\documentclass[12pt,a4j]{jsreport}
\usepackage[dvipdfmx]{graphicx} % Overleafではdvipdfmx不要
\usepackage[deluxe,uplatex]{otf}  % 日本語用フォントサポート
\usepackage{caption}  % キャプションの整形
\usepackage{comment} 
\usepackage[numbers,sort&compress]{natbib}

\setcitestyle{numbers,square,comma}

% 各章のタイトルが改行されてしまう現象が発生したので追記
\usepackage{titlesec}
\titleformat{\chapter}[hang]
  {\huge\bfseries}
  {第\thechapter{}章\ }
  {0pt}
  {}

\begin{document}

% タイトルページ
\thispagestyle{empty}
\begin{center}
修士論文/ 課題研究報告書\\% どちらか一方を消してください.
\vfill
〇〇〇〇〇題目〇〇〇〇〇\\
\vfill
〇〇著者名〇〇\\
\vfill
主指導教員  〇〇 〇〇\\
\vfill
北陸先端科学技術大学院大学\\
先端科学技術研究科\\
(〇〇〇〇)\\ %取得希望学位
\vfill
令和〇〇年〇月\\ % 学位授与年月
\vfill
\end{center}

\clearpage
\thispagestyle{plain}

% Abstract
\centerline{Abstract}
技能継承は高齢化社会の重要課題である. 従来, 技能継承は主に教科書やマニュアルを通じて行われてきた. しかし, これらの文書だけでは熟練技能者が長年の経験から獲得した暗黙知や微妙なコツを十分に伝えきれないという問題がある. この問題を解決するために熟練者の暗黙知を様々な角度から抽出可能な知識構造化という手法がある. 本研究では, 大規模言語モデル(LLM)を活用し, チャット履歴を自動解析して新たな技能要素を抽出し, 知識構造化を支援するシステムを提案する. 本システムは既存の動作モデルに対し, どの時点で手順やコツを追加・修正すべきかを提案する. これにより, 熟練者の経験に基づく動作の技能を効率的に表出化し, 学習者の技能習得を促進することが期待される. 本研究では社交ダンスを例として大規模言語モデルの可能性を検証する. 
\clearpage

% 目次
\tableofcontents
\clearpage

% 図目次と表目次
\listoffigures
\clearpage
\listoftables
\clearpage

% ページ番号をリセット
\setcounter{page}{1}

% 第1章{序論}
\chapter{はじめに}
 熟練技能者の高齢化や後継者不足が深刻化する中,技能伝承が課題になっている. 労働政策研究・研修機構の調査によれば, 2020年の時点で調査対象の企業のうち技能継承を重要だと考えている企業が95\%に達している. しかしながら, 技能者の人材育成や能力開発の取り組みがうまくいっていると認識している企業は約55\%にとどまっている \cite{JILPT2020}. 
 技能伝承を実現するためには, 熟練技能者が無意識のうちに身につけている技能やノウハウ, すなわち暗黙知を表出し, それを育成対象の人材にわかりやすく共有する必要がある.
 熟練技能者の暗黙知を表出する手法として, インタビューベースの方法が用いられている\cite{Onozato1998, Yashiro2021,Ogawa2011}. 

% 第2章{先行研究・関連研究}
\chapter{関連研究}
\section{セクション名}
\subsection{サブセクション名}
〇〇〇〇〇〇〇〇〇〇〇〇〇〇〇〇〇〇

% 第3章{提案手法}
\chapter{研究手法}
〇〇〇〇〇〇〇〇〇〇〇〇〇〇〇〇〇〇 (図 \ref{fig1})

% 第4章{実験}
\chapter{実験・事例}
\begin{table}[h]
\centering
\begin{tabular}{r|rr}
& a & b\\ \hline
1& 0.25 & 0.33\\
2& 0.75 & 0.66\\
\end{tabular}
\caption{表のキャプション}\label{table1}
\end{table}

% 第5章{結果}
\chapter{結果}
xxxxxxxxxxxxxxxx


% 第6章{考察}
\chapter{考察}
xxxxxxxxxxxxxxx

% 第7章{まとめ}
\chapter{結論}

% 参考文献
\renewcommand{\bibname}{参考文献}
\bibliographystyle{unsrtnat}
\bibliography{Master-paper}

% 付録
%% 付録用の章見出しの設定
\newcommand{\appchapterfmt}{
  \titleformat{\chapter}[hang]
    {\huge\bfseries}
    {付録~\Alph{chapter}\ }
    {0pt}
    {}
}\setcounter{chapter}{0} % 章番号をリセット
\appchapterfmt  % 付録用の章見出しを適用
%% 付録開始
\appendix
\chapter{資料}
xxxxxxxxxxxxxxxxxx

\begin{comment}

\end{comment}

\end{document}
